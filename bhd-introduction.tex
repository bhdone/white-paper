\chapter{BHD1 Introduction}
BHD1 is a new crypto currency based on the CPoS(Conditioned Proof of Space) mechanism. By using hard disk as a consensus participant, it can significantly lower energy consumption and entry barrier, making mining of crypto currency safer, more decentralized and for everyone. BHD1 generates its unique value through mathematics and code. This White Paper will explain and elaborate on the monetary and technical attributes of BHD1.
\section{Crypto Currency}
\begin{flushleft}
    When it comes to crypto currency, before the well-known Bitcoin, the entire crypto community has begun to experiment on a better international payment channel, such as Dai-Wei's Ripple and B-Money.
\end{flushleft}
\begin{flushleft}
    Ripple has been used in the settlement between banks in different countries, but never became quite as popular as was Bitcoin, because it is considered too centralized for a crypto currency. Compared to those decentralized crypto currencies, Ripple has always been more appealing to enterprise and business users, but less to the crypto enthusiasts, because its token generation procedure does not involve or incentivize the crypto enthusiasts.
\end{flushleft}
\begin{flushleft}
    B-Money causes network congestion due to the need for network synchronization in its design. At that time, the network speed was not so fast. During the sending and receiving of currency, network lag often caused problems, sometimes user receives no reply while waiting for a network packet. The system was impractical for mass adoption.
\end{flushleft}
\begin{flushleft}
    Then Bitcoin came to stage with its own Nakamoto consensus, which is the asynchronous PoW consensus. In the early days, no one was optimistic about this project. The consensus did not use simultaneous transactions to ensure that transaction results are right, but instead adopted a very interesting mechanism: the longest chain. That is to say, in this distributed system, the nodes manipulates packages and composes the chain, which includes the transaction with the correct result. For this specific package to appear, of course, the nodes in this system have to jointly verify. Only given a timeout package and only when more people participate in the accreditation before timeout, will this package reach consensus.
\end{flushleft}
\begin{flushleft}
    In this system, there is a situation where nodes can collectively do bad things, so that the correct transaction is not packaged, and the transmission of the network is invalid. Since asynchronous system avoids excessive communication in the network, it is more suitable for multiple-step transactions, While the risk of this mechanism lies with the possibility of the majority of CPU power being controlled by dishonest nodes. A good example is the later appeared $51\%$ double spend attack. The last thing the financial system should do is to roll back or double spend, that is also why Bitcoin was not widely accepted at the beginning.
\end{flushleft}
\begin{flushleft}
    Over time, a lot of participants joined the system for the financial benefits. Since the difficulty (for manipulating package mentioned above) of the system raised, the cost of harassing the working system has greatly increased for the bad guys. More stable the system, more profitable being honest rather than being dishonest. At this time, people began to realize the fascination of this crypto currency and numerous fans appeared. After years of difficulty increase, the BTC system has gradually stabilized, making it much harder to do double-spend or rollback. It also inspired the original teachings of Bitcoin, and gathered many crypto enthusiasts. It also inspired the original teachings of Bitcoin, and gathered many crypto enthusiasts. At this time, several new types of crypto currencies had been born or made by forks and copies, many got attacked because of their low computational power. The systems with low difficulty are unsafe and can be easily attacked, while highly available systems need tremendous energy consumption.
\end{flushleft}
\begin{flushleft}
    We can put a summary on Bitcoin tech features below.
\end{flushleft}
\begin{flushleft}
    Bitcoin was never aggressive on using new tech, but chose to adopt relatively mature technologies to build a safe and reliable Peer-to-Peer cash system.The more validated and simple the technology is, the more secure and trustworthy the system will be. For example, the SHA256 algorithm in Nakamoto consensus, is designed by NSA (US National Security Agency), with proven reliability. It seems that the initial design never considered the current ASIC (Application-Specific Integrated Circuit) and power monopoly issues, but focused on pursuing ultimate system security, even sacrificed some of the high efficiency or high concurrency features of internet.
\end{flushleft}
\section{Seeking Alternatives}
\begin{flushleft}
    When numerous resources are being used in the mining procedure and costs are gradually increasing, crypto currency enthusiasts have started looking for alternatives to lower power consumption in two different ways: either using new consensus to lower energy cost or using more general apparatus to lower the cost of mass production. The golden age of ASIC mining device and anti-ASIC algorithm implementation had come. The original intention of Ethereum and Monero was to resist ASIC, using a different non-ASIC-friendly consensus to keep the system away from ASIC manufacturers’ manipulation while keeping the energy consumption low. However after a period of time, ASIC manufacturers still found ways to design devices that would work with the corresponding algorithm. Among those ASIC ones, Litecoin has to be mentioned. It started with Scrypt which is an anti-ASIC mining algorithm, and soon ASIC manufacturers started producing ASIC mining devices that could work with Scrypt.
\end{flushleft}
\begin{flushleft}
    BHD1 provides the perfect solution for the issues mentioned above. It brings a method for crypto zealot to make general apparatus while keeping the energy consumption low. Meanwhile, BHD1 maintains a relatively high difficulty level to ensure the stability of the system by using its consensus Proof of Space (abbr. PoS). The PoS consensus used by BHD1 is also one of the most decentralized consensus mechanisms in this era. Compared with the PoW, where hash power rules, the PoS consensus is ruled by storage power, but slightly different from the cloud storage. PoS utilizes hard disks as a more economical consensus method, so that more people can participate in construction of the system-stabilizing hash power with their own devices. It was the original intention of Nakamoto to design PoW, a decentralized system and an innovative path to real decentralization for everyone, raising consciousness in every new comer to think about and overturn the existing design. BHD1 has inherited BTC’s spirit, now the new PoS mechanism is responsible for bringing a better future for crypto currency, and engaging more people in the construction of the economic system.
\end{flushleft}
